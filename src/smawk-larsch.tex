<<<<<<< Updated upstream
\documentclass[
  lualatex,
  ja=standard,
  compress,
  hyperref={colorlinks, urlcolor=magenta, linkcolor=blue!55!black},
  dvipsnames,
  svgnames,
]{beamer}

\newcommand{\xgets}[1]{\xleftarrow{#1}}
\newcommand{\wildcard}{*}
\newcommand{\DP}{\mathrm{dp}}
\newcommand{\Ei}[1]{\operatorname{Ei}(#1)}
\newcommand{\li}[1]{\operatorname{li}(#1)}
\newcommand{\Li}[1]{\operatorname{Li}(#1)}
\newcommand{\IntC}{\text{\textcolor{gray}{const}}}
\newcommand{\lpf}[1]{\mathrm{lpf}(#1)}
\newcommand{\gpf}[1]{\mathrm{gpf}(#1)}

\usepackage{meowbeamer}

\begin{luacode*}  
  require("lua/latest-commit")

  get_latest_commit = latest_commit.get_latest_commit
  commit_hash_and_date = latest_commit.commit_hash_and_date
\end{luacode*}

\title{\textsc{Smawk} \& \textsc{Larsch} Algorithm}
\author[えびちゃん]{えびちゃん(\hspace{.05em}\textcolor{useryellow}{\lato{\textbf{rsk0315}}}\hspace{.05em})}
\date{Sep.~24, 2022 @ねこねこ勉強ぱーてぃ\\*[.5em]
  {\footnotesize 更新:}\directlua{commit_hash_and_date("\jobname.tex")}}

\begin{document}

\begin{frame}
  \maketitle
\end{frame}

\section{導入}
\begin{frame}
  \frametitle{目標}

  \begin{itemize}
  \item 用語を理解する
    \begin{itemize}
    \item Monge性
    \item (concave|convex) QI
    \item (concave|convex) totally monotone
    \end{itemize}
  \item 以下で定まる$\DP[1], \dots, \DP[n-1]$を$O(n)$時間で求める
    \begin{itemize}
    \item $\DP[i] = \min_{j\in\halfco{0}{i}} a_j + f(j, i)$
      % dp[i + 1][j] = min dp[i][j'] + f(j', j)
    \item $\DP[i] = \min_{j\in\halfco{0}{i}} \DP[j] + f(j, i)$
      % dp[i + 1][j] = min dp[i + 1][j'] + f(j', j)
      \begin{itemize}
      \item 上記二つにおいて、$f$はQIを満たすとする。
      \item $\DP[0]$は与えられるとする。
      \end{itemize}
    \end{itemize}
  \end{itemize}
\end{frame}

\begin{frame}
  \frametitle{解ける問題の例 \theslidetopic}

  \begin{block}{EDPC Z}
    以下で定義される$\DP[n - 1]$を求めよ。ただし$a$は単調増加。
    $$
    \begin{aligned}
    \DP[0] &= 0, \\
    \DP[i] &= \min_{j\in\halfco{0}{i}} (a_i-a_j)^2+c. \\
    \end{aligned}
    $$
  \end{block}
  以下のように変形する:
  $$
  \DP[i] = \min_{j\in\halfco{0}{i}} \left\{a_j\cdot(-2a_i)+a_j^2\right\}+(a_i^2+c).
  $$
  直線$y = a_j\cdot x + a_j^2$たちの$x = -2a_i$での最小値に帰着できる。

  これは、convex hull trickで$O(n)$時間で解ける。
\end{frame}

\begin{frame}
  \frametitle{解ける問題の例 \theslidetopic}

  \begin{block}{EDPC Z+}
    以下で定義される$\DP[n - 1]$を求めよ。ただし$a$は単調増加。
    $$
    \begin{aligned}
    \DP[0] &= 0, \\
    \DP[i] &= \min_{j\in\halfco{0}{i}} (a_i-a_j)^3+c. \\
    \end{aligned}
    $$
  \end{block}
  以下のように変形する(?):
  $$
  \DP[i] = \min_{j\in\halfco{0}{i}} \left\{a_j\cdot (-3a_i^2)+a_j^2\cdot 3a_i-a_j^3\right\}+(a_i^3+c).
  $$
  これは、convex hull trickでは難しそう。→ \textsc{Larsch}なら解ける。
\end{frame}

\section{用語}
\begin{frame}
  \frametitle{Monge性}
\end{frame}

\begin{frame}
  \frametitle{Quadrangle Inequality}
\end{frame}

\begin{frame}
  \frametitle{Monotonicity}
\end{frame}

\begin{frame}
  \frametitle{Totally Monotonicity}
\end{frame}

\section{\textsc{Smawk}}
\begin{frame}
  \frametitle{問題設定}
\end{frame}

\section{\textsc{Larsch}}
\begin{frame}
  \frametitle{問題設定}
\end{frame}

\section{おわり}
\begin{frame}
  \vspace{3em}

  \Thankyou
\end{frame}

\end{document}
||||||| constructed merge base
=======
\documentclass[
  lualatex,
  ja=standard,
  compress,
  hyperref={colorlinks, urlcolor=magenta, linkcolor=blue!55!black},
  dvipsnames,
  svgnames,
]{beamer}

\newcommand{\xgets}[1]{\xleftarrow{#1}}
\newcommand{\wildcard}{*}
\newcommand{\DP}{\mathrm{dp}}
\newcommand{\Ei}[1]{\operatorname{Ei}(#1)}
\newcommand{\li}[1]{\operatorname{li}(#1)}
\newcommand{\Li}[1]{\operatorname{Li}(#1)}
\newcommand{\IntC}{\text{\textcolor{gray}{const}}}
\newcommand{\lpf}[1]{\mathrm{lpf}(#1)}
\newcommand{\gpf}[1]{\mathrm{gpf}(#1)}

\newcommand{\placeholder}{\bullet}

\usepackage{meowbeamer}

\begin{luacode*}  
  require("lua/latest-commit")

  get_latest_commit = latest_commit.get_latest_commit
  commit_hash_and_date = latest_commit.commit_hash_and_date
\end{luacode*}

\title{\textsc{Smawk} \& \textsc{Larsch} Algorithm}
\author[えびちゃん]{えびちゃん(\hspace{.05em}\textcolor{useryellow}{\lato{\textbf{rsk0315}}}\hspace{.05em})}
\date{Sep.~24, 2022 @ねこねこ勉強ぱーてぃ\\*[.5em]
  {\footnotesize 更新:}\directlua{commit_hash_and_date("\jobname.tex")}}

\begin{document}

\begin{frame}
  \maketitle
\end{frame}

\section{導入}
\begin{frame}
  \frametitle{目標}

  \begin{itemize}
  \item 用語を理解する
    \begin{itemize}
    \item Monge性
    \item (concave|convex) QI
    \item (concave|convex) totally monotone
    \end{itemize}
  \item 以下で定まる$\DP[1], \dots, \DP[n-1]$を$O(n)$時間で求める
    \begin{itemize}
    \item $\DP[i] = \min_{j\in\halfco{0}{i}} a_j + f(j, i)$
      % dp[i + 1][j] = min dp[i][j'] + f(j', j)
    \item $\DP[i] = \min_{j\in\halfco{0}{i}} \DP[j] + f(j, i)$
      % dp[i + 1][j] = min dp[i + 1][j'] + f(j', j)
      \begin{itemize}
      \item 上記二つにおいて、$f$はQIを満たすとする。
      \item $\DP[0]$は与えられるとする。
      \end{itemize}
    \end{itemize}
  \end{itemize}
\end{frame}

\begin{frame}
  \frametitle{解ける問題の例 \theslidetopic}

  \begin{block}{EDPC Z}
    以下で定義される$\DP[n - 1]$を求めよ。ただし$a$は単調増加。
    $$
    \begin{aligned}
    \DP[0] &= 0, \\
    \DP[i] &= \min_{j\in\halfco{0}{i}} (a_i-a_j)^2+c. \\
    \end{aligned}
    $$
  \end{block}
  以下のように変形する:
  $$
  \DP[i] = \min_{j\in\halfco{0}{i}} \left\{a_j\cdot(-2a_i)+a_j^2\right\}+(a_i^2+c).
  $$
  直線$y = a_j\cdot x + a_j^2$たちの$x = -2a_i$での最小値に帰着できる。

  これは、convex hull trickで$O(n)$時間で解ける。
\end{frame}

\begin{frame}
  \frametitle{解ける問題の例 \theslidetopic}

  \begin{block}{EDPC Z+}
    以下で定義される$\DP[n - 1]$を求めよ。ただし$a$は単調増加。
    $$
    \begin{aligned}
    \DP[0] &= 0, \\
    \DP[i] &= \min_{j\in\halfco{0}{i}} (a_i-a_j)^3+c. \\
    \end{aligned}
    $$
  \end{block}
  以下のように変形する(?):
  $$
  \DP[i] = \min_{j\in\halfco{0}{i}} \left\{a_j\cdot (-3a_i^2)+a_j^2\cdot 3a_i-a_j^3\right\}+(a_i^3+c).
  $$
  これは、convex hull trickでは難しそう。→ \textsc{Larsch}なら解ける。
\end{frame}

\section{用語}
\begin{frame}
  \frametitle{Monge性}

  \ruby{Monge}{もんじゅ}
\end{frame}

\begin{frame}
  \frametitle{Quadrangle Inequality, Concavity and Convexity}

  2変数関数$W(\placeholder, \placeholder)$が\emph{concave}であるとは次のことを言う:
  $$
  i<i'<j<j' \implies \underbrace{W(i, j) + W(i', j')}_{\circ} \le \underbrace{W(i, j') + W(i', j)}_{\star}.
  $$

  四角形状の点での値に関する不等式(quadrangle inequality)である。
%%   $$
%%   \left(\begin{matrix}
%%     \circ & \star \\
%%     \star & \circ
%%   \end{matrix}\right)
%%   $$
  \begin{figure}
    \begin{tikzpicture}
      \datavisualization[
        school book axes,
        visualize as scatter/.list = {circ, star},
        style sheet = cross marks,
        circ = {
          style = {
            mark options = {black},
            mark = o,
          },
        },
        star = {
          style = {
            mark options = {black},
            mark = star,
          },
        },
%%         yscale = -1,
        x axis = {
          max value = 3,
        },
        y axis = {
          max value = 2,
        },
      ]
      data [format=named] {
        x=1, y=2, set=circ
        x=2, y=3, set=circ
        x=1, y=3, set=star
        x=2, y=2, set=star
      };
    \end{tikzpicture}
    \caption{QI}
    
  \end{figure}
\end{frame}

\begin{frame}
  \frametitle{Monotonicity}
\end{frame}

\begin{frame}
  \frametitle{Totally Monotonicity}
\end{frame}

\section{\textsc{Smawk}}
\begin{frame}
  \frametitle{問題設定}
\end{frame}

\section{\textsc{Larsch}}
\begin{frame}
  \frametitle{問題設定}
\end{frame}

\section{おわり}
\begin{frame}
  \vspace{3em}

  \Thankyou
\end{frame}

\end{document}
>>>>>>> Stashed changes
