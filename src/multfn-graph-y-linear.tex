\documentclass[lualatex, ja=standard]{standalone}

\usepackage{luacode}

\usepackage{meow}

\usetikzlibrary{plotmarks}

\definecolor{ThomasRed}{RGB}{128,32,64}
\definecolor{ThomasBlue}{RGB}{52,52,178}
\colorlet{pro}{ThomasBlue!65!white}
\colorlet{con}{ThomasRed!65!white}

\newcommand{\xgets}[1]{\xleftarrow{#1}}
\newcommand{\DP}{\mathrm{dp}}
\newcommand{\lpf}[1]{\mathrm{lpf}(#1)}

\pgfkeys{
  /pgf/data visualization/style sheets/graph/.cd,
  1/.style={ThomasBlue},
  2/.style={ThomasRed},
  default style/.style={black}
}
\usetikzlibrary{datavisualization.formats.functions}
% https://tex.stackexchange.com/questions/281561/tikz-and-exponential-style-tick-label
\def\mytypesetter#1{%
  \newcommand{\F}{0}%
  \newcommand{\M}{1}%
  \newcommand{\E}{0}%
  \pgfmathfloatparsenumber{#1}%
  \pgfmathfloattomacro{\pgfmathresult}{\F}{\M}{\E}%
  \pgfutilensuremath{10^{\E}}%
}

\begin{document}

\begin{tikzpicture}
  % https://qiita.com/t_kemmochi/items/1aeafd5ec3f5f7edc4f2
  \datavisualization [
    scientific axes = clean,
    visualize as scatter/.list = {black, min25},
    visualize as line/.list = {black, min25},
    style sheet = cross marks,
    black = {
      style={,
        mark options={con},
        con, thin, dashed,
      },
      label in legend={text=$\Theta(n^{3/4}/\log(n))$?-time},
    },
    min25 = {
      style={,
        mark options={pro},
        pro, thin, dashed,
      },
      label in legend={text=$\Theta(n^{2/3})$-time},
    },
    x axis = {
      logarithmic,
      % https://tex.stackexchange.com/questions/281561/tikz-and-exponential-style-tick-label
      ticks and grid = {
        step = 1,
        tick typesetter/.code=\mytypesetter{##1}
      },
      ticks = {
        minor steps between steps = 9,
      },
      label = {$n$},
    },
    y axis = {
      max value = 100,
      label = {time (s)},
    }
  ]
  data [format=named] {
    x=1e9, y=15.282e-3, set=min25
    x=1e10, y=65.928e-3, set=min25
    x=1e11, y=289.12e-3, set=min25
    x=1e12, y=1.2963, set=min25
    x=1e13, y=6.1281, set=min25
    x=1e14, y=27.110, set=min25

    x=1e9, y=15.561e-3, set=black
    x=1e10, y=88.650e-3, set=black
    x=1e11, y=498.08e-3, set=black
    x=1e12, y=2.8150, set=black
    x=1e13, y=16.681, set=black
    x=1e14, y=99.172, set=black
  };
\end{tikzpicture}

\end{document}
