\documentclass[
  lualatex,
  ja=standard,
  compress,
  hyperref={colorlinks, urlcolor=magenta, linkcolor=blue!55!black},
  dvipsnames,
  svgnames,
]{beamer}

\usepackage{meowbeamer}

\newfontfamily\tsukuard{TsukushiAMaruGothic.ttc}[
  Path = ./,
  Extension = .ttc,
  UprightFeatures = {FontIndex=0},
  BoldFeatures = {FontIndex=1},
]
\newcommand{\jaen}[1]{{\tsukuard #1}}

\begin{luacode*}  
  require("lua/latest-commit")
  get_latest_commit = latest_commit.get_latest_commit
  commit_hash_and_date = latest_commit.commit_hash_and_date
\end{luacode*}

\title{機能の紹介を兼ねたサンプルのスライド}
\author[えびちゃん]{えびちゃん(\hspace{.05em}\textcolor{useryellow}{\lato{\textbf{rsk0315}}}\hspace{.05em})}
\date{Aug.~20, 2022 @ねこねこ勉強ぱーてぃ\\*[.5em]
  {\footnotesize 更新:}\directlua{commit_hash_and_date("\jobname.tex")}}

\begin{document}

\begin{frame}
  \maketitle
\end{frame}

\section{紹介}
\begin{frame}
  \frametitle{お気に入り機能 \theslidetopic}

  お気に入りの機能を載せてみる。

  \begin{itemize}
  \item 各種リンク
    \begin{itemize}
    \item ページ上部の章名・丸、下部のタイトル
    \end{itemize}
  \item ページ番号
    \begin{itemize}
    \item \texttt{\textbackslash appendix}以降は別で採番
    \end{itemize}
  \item トピックごとのページ番号
    \begin{itemize}
    \item スライド名の後ろのIやIIなどを自動採番
    \end{itemize}
  \end{itemize}
\end{frame}

\begin{frame}
  \frametitle{お気に入り機能 \theslidetopic}

  \begin{itemize}
  \item Lua\LaTeX{}なのでぱそこんに頼りやすい
    \begin{itemize}
    \item commit hashを取得してタイトルページに載せたり
    \item DPの例示のスライドで実際にDPしたり
    \item Luaが特別好きかというとそうではないが
    \end{itemize}
  \item 余白ができてしまった
    \begin{itemize}
    \item 思い浮かんだら追記する
    \end{itemize}
  \end{itemize}
\end{frame}

\setcounter{slidetopic}{0}
\begin{frame}
  \frametitle{フォント \theslidetopic}

  \begin{itemize}
  \item 和文フォント:筑紫\jaen{A}丸ゴシック
  \item 英文フォント:Junicode
    \begin{itemize}
    \item sans-serifでいいのがあれば変えるかも
    \item st ctのligatureは好きだけど和英混じりだとうるさい?
    \end{itemize}
  \item 数式フォント:AMS Euler + Computer Concrete
    \begin{itemize}
    \item 数学ガールに影響を受けている
    \end{itemize}
  \item 等幅フォント:TheSansMono Condensed
    \begin{itemize}
    \item O'Reilly本に影響を受けている
    \item お値段がそれなりにする
    \end{itemize}
  \end{itemize}
\end{frame}

\begin{frame}[fragile]
  \frametitle{フォント \theslidetopic}

  フォントの試し書き。\texttt{minted}を使うとコメントに数式を書ける。
  \begin{minted}[mathescape=true]{rust}
    // in $O(n\log(n))$ time
    for i in 1..=n {
      for j in 1..=n / i {
        // ...
      }
    }
  \end{minted}

  数式用のlorem ipsum的な概念があるとうれしい。
  $$
  f(a) = \frac{1}{2\pi i}\, \oint_{\gamma} \frac{f(z)}{z-a}\, \dd{z}.
  $$
\end{frame}

\section{おわり}
\begin{frame}
  \frametitle{おわり}

  気が向いたら何かを書く
\end{frame}

\begin{frame}
  \vspace{3em}
  \Thankyou
\end{frame}

\end{document}
